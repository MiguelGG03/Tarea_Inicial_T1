%&latex
\documentclass[a4paper]{article}

\usepackage[T1]{fontenc}
\usepackage[spanish]{babel}
\begin{document}

%+Title
\title{\textbf{Un algoritmo para reconocimiento de actividades humanas de dos fases en el ambito de la Industria 4.0 y procesos humanos dirigidos}
}

\author{Borja Bordel\(^{1}\), Ram�n Alcarria\(^{1}\), Diego S�nchez-de-Rivera\(^{1}\)}
\date{\(^{1}\)Universidad Polit�cnica de Madrid, \begin{center}Madrid, Espa�a\end{center}}
\maketitle
%-Title

\begin{}
    \textbf{Resumen: }Futuros sistemas industiales, una revoluci�n conocida como Industria 4.0, son visualizados para integrar a la gente en el mundo cibernetico como prsosumidores (proveedores de servicios y consumidores). En este contexto, los procesos impulsados por humanos aparecen como una realidad esencial y aparecen instrumentos para crear bucles de informaci�n de opiniones entre el subsistema social (personas) y el subsistema cibern�tico (componentes tecnol�gicos) son requeridos. Aunque muchos instrumentos distintos han sido propuestos, hoy en d�a las t�cnicas del patr�n de reconocimiento son las m�s prometedoras. Sin embargo, estas soluciones presentan algunos  problemas pendientes. Por ejemplo, estas t�cnicas dependen de el hardware seleccionado para adquirir informaci�n de los usuarios; o presentan un l�mite en la precisi�n del proceso de reconocimiento. Para abordar esta situaci�n, en este art�culo se propone un algor�tmo de dos fases para integrar a las personas en el sistema de la Industria 4.0 y los procesos impulsados por humanos. El algoritmo describe acciones complejas como composiciones de movimientos simples. Acciones complejas reconocidas usando los Modelos Ocultos de Markov, y los movimientos simples son reconocidos usando el Ajuste de Tiempo Din�mico. De esa manera, solo los movimientos son dependientes en el hardware de los dispositivos empleado para recolectar informaci�n, y la precisi�n del reconocimiento de acciones complejas mejora considerablemente. Una validaci�n experimental real es tambi�n llevada a cabo para evaluar y comparar el rendimiento de la soluci�n propuesta.
\begin{}\textbf{Palabras clave:} Industria 4.0; Patr�n de reconocimiento; Ajuste de Tiempo Din�mico; Inteligencia Artificial; Modelos Ocultos de Markov\end

    
    \end{abstract}
%-Abstract

%+Contents

%-Contents

\section{Introducci�n}
This template provides a sample layout of a Standard \LaTeX{} Article.

The front matter has a number of sample entries that you should replace
with your own. 

\section{Document Class Options}
The typesetting specification selected by this document template
uses the default class options. There are a number of class options 
supported by this document class. The available options include 
setting the paper size, the point size of the font used in the 
document body and others.

\subsection{Customizing Class Options}
Select `Insert', `Document Properties ...', the `Generic' tab
and then modify desired class options in appeared dialog.
Changes will be applied after pressing the 'OK' button.

%+Bibliography
\begin{thebibliography}{99}
\bibitem{Label1} ...
\bibitem{Label2} ...
\end{thebibliography}
%-Bibliography

\end{document}


